%
% A simple LaTeX template for Books
%  (c) Aleksander Morgado <aleksander@es.gnu.org>
%  Released into public domain
%

\documentclass{book}
\usepackage[a4paper, top=3cm, bottom=3cm]{geometry}
\usepackage[latin1]{inputenc}
\usepackage{setspace}
\usepackage{fancyhdr}
\usepackage{tocloft}


\usepackage{listings}
\usepackage{color}

\definecolor{mygreen}{rgb}{0,0.6,0}
\definecolor{mygray}{rgb}{0.5,0.5,0.5}
\definecolor{mymauve}{rgb}{0.58,0,0.82}
\lstset{ %
  backgroundcolor=\color{white},   % choose the background color; you must add \usepackage{color} or \usepackage{xcolor}
  basicstyle=\footnotesize\ttfamily,        % the size of the fonts that are
  % used for the code
  breakatwhitespace=false,         % sets if automatic breaks should only happen at whitespace
  breaklines=true,                 % sets automatic line breaking
  captionpos=b,                    % sets the caption-position to bottom
  commentstyle=\color{mygreen},    % comment style
  deletekeywords={...},            % if you want to delete keywords from the given language
  escapeinside={\%*}{*)},          % if you want to add LaTeX within your code
  extendedchars=true,              % lets you use non-ASCII characters; for 8-bits encodings only, does not work with UTF-8
  keepspaces=true,                 % keeps spaces in text, useful for keeping indentation of code (possibly needs columns=flexible)
  keywordstyle=\color{blue},       % keyword style
  language=C,                 % the language of the code
  otherkeywords={*,...},            % if you want to add more keywords to the set
  numbers=left,                    % where to put the line-numbers; possible values are (none, left, right)
  numbersep=1pt,                   % how far the line-numbers are from the code
  numberstyle=\tiny\color{mygray}, % the style that is used for the line-numbers
  rulecolor=\color{black},         % if not set, the frame-color may be changed on line-breaks within not-black text (e.g. comments (green here))
  showspaces=false,                % show spaces everywhere adding particular underscores; it overrides 'showstringspaces'
  showstringspaces=false,          % underline spaces within strings only
  showtabs=false,                  % show tabs within strings adding particular underscores
  stepnumber=1,                    % the step between two line-numbers. If it's
  % 1, each line will be numbered
  stringstyle=\color{mymauve},     % string literal style
  tabsize=1,                       % sets default tabsize to 2 spaces
  title=\lstname                   % show the filename of files included with \lstinputlisting; also try caption instead of title
}

\begin{document}


\pagestyle{empty}
%\pagenumbering{}
% Set book title
\title{\textbf{ LET'S BUILD A COMPILER!}}
% Include Author name and Copyright holder name
\author{Jack W. Crenshaw}



% 1st page for the Title
%-------------------------------------------------------------------------------
\maketitle


% 2nd page, thanks message
%-------------------------------------------------------------------------------
\thispagestyle{empty}
\newpage



% General definitions for all Chapters
%-------------------------------------------------------------------------------

% Define Page style for all chapters
\pagestyle{fancy}
% Delete the current section for header and footer
\fancyhf{}
% Set custom header
\lhead[]{\thepage}
\rhead[\thepage]{}

% Set arabic (1,2,3...) page numbering
\pagenumbering{arabic}

% Set double spacing for the text
\doublespacing



% Not enumerated chapter
%-------------------------------------------------------------------------------
\chapter*{Introduction}
\input{introduction.tex}

% If the chapter ends in an odd page, you may want to skip having the page
%  number in the empty page
\newpage
\thispagestyle{empty}



% First enumerated chapter
%-------------------------------------------------------------------------------
\chapter{Chapter 1}
\subsection{Getting Started}
If you've read the introduction document to this series, you will
already know what  we're  about.    You will also have copied the
cradle software  into your Turbo Pascal system, and have compiled
it.  So you should be ready to go.


The purpose of this article is for us to learn  how  to parse and
translate mathematical expressions.  What we would like to see as
output is a series of assembler-language statements  that perform
the desired actions.    For purposes of definition, an expression
is the right-hand side of an equation, as in
\[
	x = 2y + \frac{3}{4z}
\]

In the early going, I'll be taking things in \textbf{VERY}  small steps.
That's  so  that  the beginners among you won't get totally lost.
There are also  some  very  good  lessons to be learned early on,
that will serve us well later.  For the more experienced readers:
bear with me.  We'll get rolling soon enough.

\subsubsection{Single Digits}
In keeping with the whole theme of this series (KISS, remember?),
let's start with the absolutely most simple case we can think of.
That, to me, is an expression consisting of a single digit.

Before starting to code, make sure you have a  baseline  copy  of
the  "cradle" that I gave last time.  We'll be using it again for
other experiments.  Then add this code:

\begin{lstlisting}[language=C]

\end{lstlisting}



% Last pages for ToC
%-------------------------------------------------------------------------------
\newpage
% Include dots between chapter name and page number
\renewcommand{\cftchapdotsep}{\cftdotsep}
%Finally, include the ToC
\tableofcontents




\end{document}
