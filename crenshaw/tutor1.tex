\subsection{Getting Started}
If you've read the introduction document to this series, you will
already know what  we're  about.    You will also have copied the
cradle software  into your Turbo Pascal system, and have compiled
it.  So you should be ready to go.


The purpose of this article is for us to learn  how  to parse and
translate mathematical expressions.  What we would like to see as
output is a series of assembler-language statements  that perform
the desired actions.    For purposes of definition, an expression
is the right-hand side of an equation, as in
\[
	x = 2y + \frac{3}{4z}
\]

In the early going, I'll be taking things in \textbf{VERY}  small steps.
That's  so  that  the beginners among you won't get totally lost.
There are also  some  very  good  lessons to be learned early on,
that will serve us well later.  For the more experienced readers:
bear with me.  We'll get rolling soon enough.

\subsubsection{Single Digits}
In keeping with the whole theme of this series (KISS, remember?),
let's start with the absolutely most simple case we can think of.
That, to me, is an expression consisting of a single digit.

Before starting to code, make sure you have a  baseline  copy  of
the  "cradle" that I gave last time.  We'll be using it again for
other experiments.  Then add this code:

\begin{lstlisting}[language=C]
//file cradle.c
void Expression(){
	sprintf(tmp,"MOVE #%c,D0",GetNumber()); 
	EmitLn(tmp); 
}

//file cradle.h
void Expression();
\end{lstlisting}


\begin{lstlisting}[language=C]
//file main.c
#include "cradle.h"
void main(){
	Init();
	Expression();
	return;
}

\end{lstlisting}

Compile\footnote{gcc -o cradle cradle.h cradle.c main.c} and run the program.
 Try any single-digit number as input. You
should get a single line of assembler-language output. Now try
any other character as input, and you'll see that the parser
properly reports an error. \textbf{Congratulation}!  You have just written a working translator!
OK, I grant you that it's pretty limited. But don't brush it off
too lightly. This little "compiler" does, on a very limited scale, exactly what any larger compiler does: it correctly
recognizes legal statements in the input "language" that we have
defined for it, and it produces correct, executable assembler
code, suitable for assembling into object format. Just as
importantly, it correctly recognizes statements that are NOT
legal, and gives a meaningful error message. Who could ask for
more? As we expand our parser, we'd better make sure those two
characteristics always hold true.
There are some other features of this tiny program worth
mentioning. First, you can see that we don't separate code
generation from parsing ... as soon as the parser knows what we
want done, it generates the object code directly. In a real
compiler, of course, the reads in GetChar would be from a disk
file, and the writes to another disk file, but this way is much
easier to deal with while we're experimenting.
Also note that an expression must leave a result somewhere. I've
chosen the 68000 register DO. I could have made some other
choices, but this one makes sense.

\subsection{Binary Expression}
Now that we have that under our belt, let's branch out a bit.
Admittedly, an "expression" consisting of only one character is
not going to meet our needs for long, so let's see what we can do
to extend it. Suppose we want to handle expressions of the form:
\begin{itemize}
  \item \(1+2\)
  \item \(4+3\)
  \item or, in general, term +/- term (That's a bit of Backus-Naur Form, or
  BNF.)
\end{itemize}

o do this we need a procedure that recognizes a term and leaves
its result somewhere, and another that recognizes and
distinguishes between a '+' and a '-' and generates the
appropriate code. But if Expression is going to leave its result
in DO, where should Term leave its result? Answer: the same
place. We're going to have to save the first result of Term
somewhere before we get the next one.
OK, basically what we want to do is have procedure Term do what
Expression was doing before. So just \textbf{RENAME function Expression}
as Term, and enter the following new version of Expression:

\begin{lstlisting}[language=C]
//file cradle.h
void Expression();
void Term();

void Add();
void Subtract();

//file cradle.c
//Old code for expression becomes Term function
void Add(){
Match('+');
	Term();
	EmitLn("ADD D0,D1");
}

void Subtract(){
	Match('-');
	Term();
	EmitLn("SUB D1,D0");
}

void Term(){
	sprintf(tmp,"MOVE #%c,D0",GetNumber()); 
	EmitLn(tmp); 
}

void Expression(){
	Term();	
	sprintf(tmp,"MOVE DO,D1");
	EmitLn(tmp);
	switch (Look){
		case '+' :
			Add();
			break;
		case '-' :
			Subtract();
			break;	
		default:
			Expected("Binary Operator");
			break;		
	}
}
\end{lstlisting}
Now run the program. Try any combination you can think of of two
single digits, separated by a '+' or a '-'. You should get a
series of four assembler-language instructions out of each run.
Now try some expressions with deliberate errors in them. Does
the parser catch the errors?

Take a look at the object code generated. There are two
observations we can make. First, the code generated is NOT what
we would write ourselves. The sequence

\begin{lstlisting}[language=Assembler]
 MOVE #n,D0
 MOVE D0,D1
\end{lstlisting}
is inefficient. If we were writing this code by hand, we would
probably just load the data directly to D1.

There is a message here: code generated by our parser is less
efficient than the code we would write by hand. Get used to it.
That's going to be true throughout this series. It's true of all
compilers to some extent. Computer scientists have devoted whole
lifetimes to the issue of code optimization, and there are indeed
things that can be done to improve the quality of code output.
Some compilers do quite well, but there is a heavy price to pay
in complexity, and it's a losing battle anyway ... there will
probably never come a time when a good assembler-language programmer
can't out-program a compiler. Before this session is
over, I'll briefly mention some ways that we can do a little optimization,
just to show you that we can indeed improve things
without too much trouble. But remember, we're here to learn, not
to see how tight we can make the object code. For now, and
really throughout this series of articles, we'll studiously
ignore optimization and concentrate on getting out code that
works.
Speaking of which: ours DOESN'T! The code is \textbf{WRONG!} As things
are working now, the subtraction process subtracts D1 (which has
the FIRST argument in it) from D0 (which has the second). That's
the wrong way, so we end up with the wrong sign for the result.
So let's fix up procedure Subtract with a sign-changer, so that
it reads
\begin{lstlisting}[language=C]
void Subtract(){
	Match('-');
	Term();
	EmitLn("SUB D1,D0");
	EmitLn("NEGATE D0");
}
\end{lstlisting}
Now our code is even less efficient, but at least it gives the
right answer! Unfortunately, the rules that give the meaning of
math expressions require that the terms in an expression come out
in an inconvenient order for us. Again, this is just one of
those facts of life you learn to live with. This one will come
back to haunt us when we get to division.
OK, at this point we have a parser that can recognize the sum or
difference of two digits. Earlier, we could only recognize a
single digit. But real expressions can have either form (or an
infinity of others). For kicks, go back and run the program with
the single input line '1'.
Didn't work, did it? And why should it? We just finished
telling our parser that the only kinds of expressions that are
legal are those with two terms. We must rewrite procedure
Expression to be a lot more broadminded, and this is where things
start to take the shape of a real parser.

\subsection{General Expressions}

In the REAL world, an expression can consist of one or more
terms, separated by "addops" ('+' or '-'). In BNF, this is
written
expression ::= term $[\langle$ addop $\rangle \langle$ term $\rangle ]$*

We can accomodate this definition of an expression with the
addition of a simple loop to procedure Expression:
\begin{lstlisting}[language=C]
	void Expression(){
	Term();	
	sprintf(tmp,"MOVE DO,D1");
	EmitLn(tmp);
	while(Look=='+' || Look == '-'){	
		switch (Look){
				case '+' :
					Add();
					break;
				case '-' :
					Subtract();
					break;	
				default:
					Expected("Binary Operator");
					break;		
			}
	}
}
\end{lstlisting}
NOW we're getting somewhere! This version handles any number of
terms, and it only cost us two extra lines of code. As we go on,
you'll discover that this is characteristic of top-down parsers
... it only takes a few lines of code to accomodate extensions to
the language. That's what makes our incremental approach
possible. Notice, too, how well the code of procedure Expression
matches the BNF definition. That, too, is characteristic of the
method. As you get proficient in the approach, you'll find that
you can turn BNF into parser code just about as fast as you can
type!
OK, compile the new version of our parser, and give it a try. As
usual, verify that the "compiler" can handle any legal
expression, and will give a meaningful error message for an
illegal one. Neat, eh? You might note that in our test version,
any error message comes out sort of buried in whatever code had
already been generated. But remember, that's just because we are
using the CRT as our "output file" for this series of
experiments. In a production version, the two outputs would be
separated ... one to the output file, and one to the screen.

\subsection{Using The Stack}
At this point I'm going to violate my rule that we don't
introduce any complexity until it's absolutely necessary, long
enough to point out a problem with the code we're generating. As
things stand now, the parser uses D0 for the "primary" register,
and D1 as a place to store the partial sum. That works fine for
now, because as long as we deal with only the "addops" '+' and
'-', any new term can be added in as soon as it is found. But in
general that isn't true. Consider, for example, the expression
\[
	1+(2-(3+(4-5)))
\]

If we put the '1' in D1, where do we put the '2'? Since a
general expression can have any degree of complexity, we're going
to run out of registers fast!

Fortunately, there's a simple solution. Like every modern
microprocessor, the 68000 has a stack, which is the perfect place
to save a variable number of items. So instead of moving the term
in D0 to D1, let's just push it onto the stack. For the benefit
of those unfamiliar with 68000 assembler language, a push is
written $-(SP)$ and a pop $(SP)+$.

So let's change the EmitLn in Expression to read:
\begin{lstlisting}[language=C]
	sprintf(tmp,"MOVE DO,-(SP)");
	EmitLn(tmp);
\end{lstlisting}
and the two lined in Add and Subtract to
\begin{lstlisting}[language=C]
//add
	sprintf(tmp,"ADD +(SP),D0");
	EmitLn(tmp);
//subtract
sprintf(tmp,"SUB +(SP),D0");
	EmitLn(tmp);	
\end{lstlisting}
Now try the parser again and make sure we haven't
broken it. Once again, the generated code is less efficient than before, but
it's a necessary step, as you'll see.

\subsection{Multiplication and Division}
Now let's get down to some REALLY serious business. As you all
know, there are other math operators than "addops" ...
expressions can also have multiply and divide operations. You
also know that there is an implied operator PRECEDENCE, or
hierarchy, associated with expressions, so that in an expression
like $2+3*4$ we know that we're supposed to multiply FIRST, then add. (See
why we needed the stack?)

In the early days of compiler technology, people used some rather
complex techniques to insure that the operator precedence rules
were obeyed. It turns out, though, that none of this is
necessary ... the rules can be accommodated quite nicely by our
top-down parsing technique. Up till now, the only form that
we've considered for a term is that of a single decimal digit.
More generally, we can define a term as a PRODUCT of FACTORS;
i.e.,


expression ::= factor $[\langle$ multop $\rangle \langle$ factor $\rangle ]*$


What is a factor? For now, it's what a term used to be ... a
single digit.

Notice the symmetry: a term has the same form as an expression.
As a matter of fact, we can add to our parser with a little
judicious copying and renaming. But to avoid confusion, the
listing below is the complete set of parsing routines. (Note the
way we handle the reversal of operands in Divide.)

\begin{lstlisting}[language=C]
	
	#include "cradle.h"
	#include <stdlib.h>
	#include <stdio.h>
	
	void Factor(){
		sprintf(tmp,"MOVE #%c,D0",GetNumber());
		EmitLn(tmp);
	}
	
	
	void Multiply(){
		Match('*');
		Factor();
		EmitLn("MULS (SP)+,D1");
	}
	
	void Divide(){
		Match('/');
		Factor();
		EmitLn("DIVS (SP)+,D1");
	}
	
	
	
	void Add(){
	Match('+');
		Term();
		EmitLn("ADD (SP)+,D1");
	}
	
	void Subtract(){
		Match('-');
		Term();
		EmitLn("SUB (SP)+,D0");
		EmitLn("NEG D0");
	}
	
	
	void Expression(){
		Term();	
		sprintf(tmp,"MOVE DO,-(SP)");
		EmitLn(tmp);
		while(Look=='+' || Look == '-'){	
			switch (Look){
					case '+' :
						Add();
						break;
					case '-' :
						Subtract();
						break;	
					default:
						Expected("Binary Operator");
						break;		
				}
		}
	}
	
	
	void Term(){
		Factor();
		while(Look=='*' || Look == '/'){
			sprintf(tmp,"MOVE D0,-(SP)");
			EmitLn(tmp);
			switch (Look){
				case '*' :
					Multiply();
					break;
				case '/' :
					Divide();
					break;	
				default:
					Expected("Mult/Div Operator");
					break;		
					}
				}
	
	}
	
	void GetChar(){
		Look = getchar();
	}
	
	void Error(char* s){
		printf("\nError: %s",s);
	}
	
	void Abort(char* s){
		printf("\nError: %s",s);
		exit(1);
	}
	
	void Expected(char* s){
		sprintf(tmp,"%s Expected",s);
		Abort(tmp);
	}
	void Match(char x){
		if(Look==x){
			GetChar();
		}else{
			sprintf(tmp,"'%c'",x);
			Expected(tmp);
		}
	}
	
	int isAlpha(char c){
		char cu = toupper(c);
		return (cu >= 'A') && (cu <= 'z');
	}
	
	int isDigit(char c){
		return (c >= '0') && (c <= '9');
	}
	
	void Emit(char* s){
	printf("\t%s",s);
	}
	
	void EmitLn(char* s){
		Emit(s);
		printf("\n");
	}
	
	char GetName(){
		char c=Look;	
		if(!isAlpha(Look)){
			Expected("Name");
		}
		
		GetChar();
		return c;
	}
	
	char GetNumber(){
		char c=Look;	
		if(!isDigit(Look)){
			Expected("Integer");
		}
		
		GetChar();
		return c;
	}
	
	
	void Init(){
		GetChar();	
	}
\end{lstlisting}