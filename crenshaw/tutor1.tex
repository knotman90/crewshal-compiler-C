\subsection{Getting Started}
If you've read the introduction document to this series, you will
already know what  we're  about.    You will also have copied the
cradle software  into your Turbo Pascal system, and have compiled
it.  So you should be ready to go.


The purpose of this article is for us to learn  how  to parse and
translate mathematical expressions.  What we would like to see as
output is a series of assembler-language statements  that perform
the desired actions.    For purposes of definition, an expression
is the right-hand side of an equation, as in
\[
	x = 2y + \frac{3}{4z}
\]

In the early going, I'll be taking things in \textbf{VERY}  small steps.
That's  so  that  the beginners among you won't get totally lost.
There are also  some  very  good  lessons to be learned early on,
that will serve us well later.  For the more experienced readers:
bear with me.  We'll get rolling soon enough.

\subsubsection{Single Digits}
In keeping with the whole theme of this series (KISS, remember?),
let's start with the absolutely most simple case we can think of.
That, to me, is an expression consisting of a single digit.

Before starting to code, make sure you have a  baseline  copy  of
the  "cradle" that I gave last time.  We'll be using it again for
other experiments.  Then add this code:

\begin{lstlisting}[language=C]

\end{lstlisting}